\chapter{TRÌNH BÀY BÀI TOÁN}

\section{Giới thiệu bài toán}
NFT Marketplace là một nền tảng giao dịch trực tuyến dựa trên blockchain, cho phép người dùng tạo, mua, bán và quản lý các tài sản kỹ thuật số độc nhất (NFT). Dự án giải quyết nhu cầu sở hữu và giao dịch tài sản kỹ thuật số một cách an toàn, minh bạch và phi tập trung. Đây không chỉ là một thị trường giao dịch mà còn là cầu nối giữa nghệ sĩ số và nhà sưu tập, góp phần nâng cao giá trị của các tác phẩm nghệ thuật và vật phẩm kỹ thuật số.

\section{Xác định các tác nhân}
\begin{itemize}
    \item \textbf{Người dùng (User):} Bao gồm nghệ sĩ số, nhà sưu tập, nhà đầu tư và người mới bắt đầu. Họ có thể tạo, mua, bán và quản lý NFT. Mỗi người dùng có một ví điện tử riêng để thực hiện các giao dịch trên blockchain.
    \item \textbf{Smart Contract Owner:} Người sở hữu hợp đồng thông minh, có quyền cập nhật phí listing khi cần thiết, nhưng không quản lý trực tiếp người dùng. Vai trò này đảm bảo tính minh bạch và phi tập trung của hệ thống.
\end{itemize}

\section{Mô tả các chức năng}
\begin{itemize}
    \item \textbf{Đối với Người dùng:}
    \begin{itemize}
        \item \textbf{Tạo NFT mới (minting):}
        \begin{itemize}
            \item Tải lên hình ảnh, video hoặc tệp kỹ thuật số
            \item Thêm thông tin mô tả, tên và các thuộc tính của NFT
            \item Thiết lập giá khởi điểm
            \item Xác nhận và thanh toán phí gas để tạo NFT
        \end{itemize}
        
        \item \textbf{Mua NFT từ người khác:}
        \begin{itemize}
            \item Xem thông tin chi tiết về NFT (giá, lịch sử, chủ sở hữu)
            \item Mua trực tiếp với giá đã niêm yết
            \item Xác nhận giao dịch và thanh toán qua ví điện tử
            \item Nhận quyền sở hữu NFT sau khi giao dịch hoàn tất
        \end{itemize}
        
        \item \textbf{Bán lại NFT đã sở hữu:}
        \begin{itemize}
            \item Chọn NFT muốn bán từ bộ sưu tập cá nhân
            \item Thiết lập giá bán cố định
            \item Xác nhận và đăng bán NFT lên marketplace
            \item Nhận thanh toán tự động khi có người mua
        \end{itemize}
        
        \item \textbf{Quản lý tài khoản và ví:}
        \begin{itemize}
            \item Kết nối và quản lý ví điện tử
            \item Xem số dư và lịch sử giao dịch
            \item Quản lý bộ sưu tập NFT cá nhân
            \item Cập nhật thông tin hồ sơ và cài đặt bảo mật
        \end{itemize}
        
        \item \textbf{Tìm kiếm và lọc NFT:}
        \begin{itemize}
            \item Tìm kiếm theo tên, nghệ sĩ, bộ sưu tập
            \item Lọc theo giá, loại tài sản, thời gian
            \item Sắp xếp theo giá, thời gian
            \item Lưu tìm kiếm và thiết lập thông báo
        \end{itemize}
        
        \item \textbf{Xem danh sách NFT trên thị trường:}
        \begin{itemize}
            \item Xem NFT đang được bán với giá cố định
            \item Theo dõi các bộ sưu tập mới và xu hướng
            \item Xem thống kê về giá và khối lượng giao dịch
        \end{itemize}
        
        \item \textbf{Theo dõi lịch sử giao dịch của NFT:}
        \begin{itemize}
            \item Xem lịch sử chuyển nhượng và giá
            \item Xem thông tin về chủ sở hữu trước đó
            \item Phân tích xu hướng giá và độ phổ biến
        \end{itemize}
    \end{itemize}
    
    \item \textbf{Đối với Smart Contract Owner:}
    \begin{itemize}
        \item \textbf{Cập nhật phí listing:}
        \begin{itemize}
            \item Điều chỉnh phí đăng bán NFT
            \item Thiết lập phí giao dịch cho marketplace
            \item Quản lý phí gas cho các hoạt động
        \end{itemize}
        
        \item \textbf{Quản lý các tham số của hợp đồng thông minh:}
        \begin{itemize}
            \item Cập nhật logic xử lý giao dịch
            \item Điều chỉnh các tham số bảo mật
            \item Quản lý danh sách token được chấp nhận
            \item Tối ưu hóa hiệu suất hợp đồng
        \end{itemize}
    \end{itemize}
\end{itemize} 